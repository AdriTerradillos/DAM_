\documentclass[12pt,a4paper]{article}

\usepackage[utf8]{inputenc}
\usepackage[T1]{fontenc}
\usepackage{geometry}
\usepackage{setspace}
\usepackage{enumitem}
\usepackage{hyperref}
\usepackage{array}
\usepackage{booktabs}
\usepackage{xcolor}

\geometry{margin=2.5cm}
\setstretch{1.25}
\hypersetup{
  colorlinks=true,
  linkcolor=blue!50!black,
  urlcolor=blue!50!black
}

\begin{document}

\title{Introducción a los Diagramas Entidad--Relación (ER) con la base de datos \textit{Sakila}}
\author{Departamento de Informática}
\date{}
\maketitle

\section*{1. Objetivo del documento}
Este texto introduce los conceptos fundamentales del modelado Entidad--Relación (ER) usando como caso real la base de datos \textbf{Sakila} (videoclub). 


\section*{Introducción a la base de datos \textit{Sakila}}

La base de datos \textbf{Sakila} es un modelo de ejemplo desarrollado por \textbf{MySQL} con fines educativos. 
Su propósito es mostrar cómo se estructura y gestiona un sistema de información realista mediante un modelo entidad–relación completo. 
El dominio elegido es el de un \textbf{videoclub internacional} que ofrece servicios de \textbf{alquiler de películas} en varias tiendas distribuidas por diferentes países.

El objetivo de Sakila no es servir como una base de datos de producción, sino como una herramienta de aprendizaje y demostración. 
Permite estudiar cómo se definen las \textbf{entidades}, las \textbf{relaciones}, las \textbf{claves primarias y foráneas}, y cómo se representan los distintos tipos de cardinalidad (1:1, 1:N, N:M). 
También resulta idónea para practicar consultas SQL, la normalización, el diseño físico y la integridad referencial.

\subsection*{1. Contexto del modelo}

Sakila reproduce de forma simplificada la operativa de un videoclub: 
los clientes alquilan películas, los empleados gestionan los pagos y las devoluciones, 
y cada tienda mantiene su propio inventario. 
Las películas están descritas con información detallada —título, año, idioma, duración, clasificación, actores y categorías— 
y se distribuyen entre distintas tiendas ubicadas en diversas ciudades y países.

El modelo está completamente normalizado e incluye los principales elementos que intervienen en el negocio:
\begin{itemize}
  \item \textbf{Clientes} y sus datos de contacto.
  \item \textbf{Empleados} y \textbf{tiendas} físicas.
  \item \textbf{Películas}, \textbf{idiomas}, \textbf{categorías} y \textbf{actores}.
  \item \textbf{Inventario}, \textbf{alquileres} y \textbf{pagos}.
\end{itemize}

\subsection*{2. Objetivo pedagógico}

Esta base de datos es ideal para introducir el \textbf{modelo Entidad–Relación (ER)} 
porque contiene todos los casos que un diseñador debe dominar:
\begin{itemize}
  \item Entidades fuertes y débiles.
  \item Relaciones de los tres tipos posibles: 1:1, 1:N y N:M.
  \item Claves primarias simples y compuestas.
  \item Tablas intermedias con atributos propios.
\end{itemize}

Gracias a esta riqueza estructural, Sakila permite observar cómo un modelo conceptual se traduce al modelo relacional 
y cómo las relaciones se implementan mediante claves foráneas.

\subsection*{3. Estructura general}

La base de datos contiene aproximadamente dieciséis tablas, organizadas en tres bloques principales:

\begin{enumerate}
  \item \textbf{Bloque de Catálogo:} contiene la información relacionada con las películas y su clasificación.  
  Incluye tablas como:  
  \texttt{film}, \texttt{actor}, \texttt{category}, \texttt{language}, \texttt{film\_actor}, \texttt{film\_category}.

  \item \textbf{Bloque de Operaciones del videoclub:} representa la actividad diaria del negocio, es decir, los alquileres, los pagos y la gestión de clientes y empleados.  
  Incluye tablas como:  
  \texttt{inventory}, \texttt{rental}, \texttt{payment}, \texttt{store}, \texttt{staff}, \texttt{customer}.

  \item \textbf{Bloque de Ubicación geográfica:} almacena la información sobre direcciones, ciudades y países, que se utiliza para vincular tiendas, clientes y empleados con sus ubicaciones.  
  Incluye tablas como:  
  \texttt{address}, \texttt{city}, \texttt{country}.
\end{enumerate}


Cada bloque está conectado con los demás mediante \textbf{claves foráneas}, 
formando una red de relaciones que permite representar la totalidad de la información necesaria 
para la gestión de un videoclub.

\section*{2. Conceptos básicos}
\subsection*{2.1 Entidad}
Una \textbf{entidad} es un tipo de objeto del mundo real sobre el que queremos guardar información. En Sakila, ejemplos de entidades son:
\texttt{film} (película), \texttt{actor}, \texttt{customer} (cliente), \texttt{store} (tienda), \texttt{rental} (alquiler), \texttt{payment} (pago), \texttt{address} (dirección), \texttt{city}, \texttt{country}, \texttt{language}.

\subsection*{2.2 Atributo}
Un \textbf{atributo} es un dato que describe a la entidad. Por ejemplo, en \texttt{film}: \texttt{title} (título), \texttt{length} (duración), \texttt{release\_year} (año), \texttt{rating}; en \texttt{actor}: \texttt{first\_name}, \texttt{last\_name}.

\subsection*{2.3 Clave primaria (PK)}
La \textbf{clave primaria} identifica \emph{de manera única} cada fila de una tabla. No puede repetirse ni ser nula.
Ejemplos: \texttt{film(film\_id)}, \texttt{actor(actor\_id)}, \texttt{customer(customer\_id)}.

\subsection*{2.4 Clave foránea (FK)}
Una \textbf{clave foránea} es un atributo en una tabla que referencia la clave primaria de otra tabla, estableciendo una relación entre ambas. Ejemplos:
\begin{itemize}
  \item \texttt{rental.customer\_id} $\rightarrow$ \texttt{customer(customer\_id)}.
  \item \texttt{inventory.film\_id} $\rightarrow$ \texttt{film(film\_id)}.
\end{itemize}

\section*{3. Entidades fuertes y débiles}
\begin{itemize}[leftmargin=1.5em]
  \item \textbf{Fuertes} (existen por sí mismas): \texttt{film}, \texttt{actor}, \texttt{category}, \texttt{customer}, \texttt{staff}, \texttt{store}, \texttt{rental}, \texttt{payment}, \texttt{address}, \texttt{city}, \texttt{country}, \texttt{language}.
  \item \textbf{Débiles / puente} (dependen de otras): \texttt{film\_actor}, \texttt{film\_category}. En Sakila, \texttt{inventory} depende lógicamente de \texttt{film} y \texttt{store} (es una copia física de una película en una tienda), aunque tenga PK propia.
\end{itemize}

\section*{4. Tipos de relaciones (cardinalidades)}
\subsection*{4.1 Relación 1:N (uno a muchos)}
Un registro de A se asocia con muchos de B, pero cada B con exactamente uno de A.
Ejemplos:
\begin{itemize}
  \item \texttt{country (1)} $\rightarrow$ \texttt{city (N)}.
  \item \texttt{city (1)} $\rightarrow$ \texttt{address (N)}.
  \item \texttt{store (1)} $\rightarrow$ \texttt{staff (N)}.
  \item \texttt{customer (1)} $\rightarrow$ \texttt{rental (N)}.
  \item \texttt{rental (1)} $\rightarrow$ \texttt{payment (N)}.
  \item \texttt{film (1)} $\rightarrow$ \texttt{inventory (N)}.
\end{itemize}

\subsection*{4.2 Relación 1:1 (uno a uno)}
Cada registro de A se asocia con \emph{exactamente uno} de B, y viceversa. En Sakila, el ejemplo didáctico es:
\begin{itemize}
  \item \texttt{store} $\leftrightarrow$ \texttt{staff} (manager): cada tienda tiene un único gerente (referenciado por \texttt{store.manager\_staff\_id}); para imponer realmente 1:1 se requiere una \textbf{restricción UNIQUE} sobre \texttt{store.manager\_staff\_id} (y coherencia con \texttt{staff.store\_id}).
\end{itemize}
\textit{Nota:} Muchos escenarios 1:1 se modelan en la práctica como 1:N con una restricción \texttt{UNIQUE} en la FK del lado N.

\subsection*{4.3 Relación N:M (muchos a muchos)}
Si una película puede tener \emph{muchos} actores y un actor puede participar en \emph{muchas} películas, no se puede guardar correctamente con una sola FK.
%
Piénsalo por un momento y convéncete de ello.

\textbf{Necesidad de tabla intermedia:}
\begin{enumerate}[label=\alph*)]
  \item Sin tabla intermedia, sólo podríamos registrar ``un actor por película'' (FK en \texttt{film}) o ``una película por actor'' (FK en \texttt{actor}), lo cual es incorrecto.
  \item La tabla intermedia (\texttt{film\_actor}) contiene \textbf{dos FKs}, una a cada entidad principal, y convierte la N:M en \textbf{dos relaciones 1:N} (\texttt{film} $\rightarrow$ \texttt{film\_actor} y \texttt{actor} $\rightarrow$ \texttt{film\_actor}).
  \item Para \textbf{evitar duplicados} (el mismo actor repetido para la misma película), la PK suele ser \textbf{compuesta} por ambas FKs:
  \[
    \texttt{PRIMARY KEY (film\_id, actor\_id)}
  \]
\end{enumerate}

\paragraph{Atributos en tablas intermedias (muy importante).}
Las tablas intermedias \emph{no sólo} guardan las FKs. \textbf{Pueden y suelen tener atributos propios}, por ejemplo, la tabla \texttt{film\_actor} representa la tabla intermedia entre \texttt{film} y \texttt{actor} y tiene estos atributos:

\begin{itemize}
  \item \texttt{film\_actor.rol} \,(\emph{principal, secundario, cameo}), 
  \item \texttt{film\_actor.orden\_facturacion} \,(posición en los créditos),
  \item \texttt{film\_actor.fecha\_contrato} \,(cuándo se incorporó),
  \item \texttt{film\_actor.cache} \,(importe pactado),
  \item \texttt{last\_update} \,(técnico; Sakila lo incluye a menudo).
\end{itemize}
En Sakila, \texttt{film\_actor} y \texttt{film\_category} ya incluyen un atributo técnico \texttt{last\_update}. En un sistema real añadiríamos los campos de negocio que correspondan a la relación.

\paragraph{Resumen de N:M en Sakila.}
\begin{itemize}
  \item \texttt{film} $\leftrightarrow$ \texttt{actor} \,$\Rightarrow$\, \texttt{film\_actor(film\_id, actor\_id, last\_update, \dots)}
  \item \texttt{film} $\leftrightarrow$ \texttt{category} \,$\Rightarrow$\, \texttt{film\_category(film\_id, category\_id, last\_update, \dots)}
\end{itemize}

\section*{5. Patrón de mapeo ER $\rightarrow$ relacional}
\begin{enumerate}
  \item \textbf{Relación 1:N} $\rightarrow$ FK en la tabla del lado N que hace referencia a la PK en la tabla del 1.
  \item \textbf{Relación 1:1} $\rightarrow$ FK con \texttt{UNIQUE} en el lado que almacena la referencia (para “forzar” 1:1).
  \item \textbf{Relación N:M} $\rightarrow$ \textbf{tabla intermedia} con FKs a ambas tablas y \textbf{PK compuesta} (o PK artificial + \texttt{UNIQUE(fk1,fk2)}).
  \item Añadir \textbf{atributos de la relación} (si existen) a la \textbf{tabla intermedia}.
%  \item Definir \textbf{acciones} de integridad (\texttt{ON DELETE/UPDATE}) de forma coherente.
%  \item \textbf{Entidad fuerte} $\rightarrow$ \textbf{tabla} con PK (normalmente un identificador numérico).
\end{enumerate}

\section*{6. Sakila: mini–mapa de algunas relaciones}
\begin{verbatim}
country (1) -> city (N) -> address (N) -> customer (N)
                               \
                                -> staff (N)
store (1) -> customer (N)
store (1) -> inventory (N) -> rental (N) -> payment (N)
film (1) -> inventory (N)
film (N) -> film_actor <- (N) actor
film (N) -> film_category <- (N) category
\end{verbatim}

\section*{7. Tarea práctica (para entregar)}
\textbf{Actividad: Genera el ER en MySQL Workbench a partir del caso Sakila} y documenta tus decisiones.

\subsection*{7.1 Pasos en MySQL Workbench}
\begin{enumerate}
  \item Abre MySQL Workbench.
	\item "Database" -> "Reverse Engineer" 
	\item Selecciona solamente las tablas.
  \end{enumerate}

\subsection*{7.2 Entregables}
\begin{itemize}
  \item Diagrama EER (captura en PDF o PNG) donde se vean claramente las tablas y las relaciones.
  \item Documento breve (1--2 páginas) respondiendo:
    \begin{enumerate}
      \item ¿Cuáles son las \textbf{claves primarias} de cada tabla?
      \item ¿Qué \textbf{claves foráneas} existen y qué relación representan (1:1, 1:N, N:M)?
      \item En las N:M, ¿qué \textbf{atributos adicionales} tienen las \textbf{tablas intermedias} y por qué?
      \item ¿Cómo se garantiza que no haya \textbf{duplicados} en la tabla intermedia?
      \item ¿Recuerdas la base de datos tienda\_online? ¿Qué relaciones N:M y 1:N tenía?
    \end{enumerate}
\end{itemize}

\bigskip
\noindent\textbf{Recordatorio clave sobre N:M:} la tabla intermedia es el \emph{lugar correcto} para poner los datos que describen la \emph{relación} (no a los extremos). Ejemplo: en \texttt{film\_actor}, el \textit{rol} o el \textit{orden de facturación} describen ``la participación de ese actor en esa película'', no al actor ni a la película por separado.

\end{document}